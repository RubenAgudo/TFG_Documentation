\chapter{Introducci\'{o}n} 
%\newpage
El Trabajo de Fin de Grado (a partir de ahora TFG) aqu\'{i} presentado fue propuesto por un profesor de la universidad.
Las razones de la aceptaci\'{o}n del TFG fueron que se ve\'{i}a una continuidad, es decir, que se pod\'{i}a 
utilizar el TFG como entrada a un grupo de investigaci\'{o}n y/o continuar con el trabajo durante un m\'{a}ster.

\section{Planteamiento del problema}
El problema que el software ULISES, dise\~nado por el grupo de investigaci\'on GaLan en colaboraci\'on
con la universidad de Navarra, intenta resolver, es ense\~nar a alumnos habilidades. Es decir, ayudar a 
instruir a las personas cosas que no son f\'acilmente evaluables mediante un examen escrito.

ULISES es una metodolog\'ia que permite unir un sistema interactivo que exista
con un sistema educativo que tambi\'en exista para poder evaluar la realizaci\'on de tareas procedimentales 
o la adquisici\'on de habilidades.

Un ejemplo de aplicaci\'on de ULISES es el software INTRASIM 
\footnote{\url{http://www.ceit.es/index.php?option=com_content&view=article&id=112&Itemid=132}}, 
en el cual el sistema interactivo es un simulador de camiones y el sistema educativo es un software
llamado DETECTIVE.

Los datos capturados en bruto, es decir, las se\~nales generadas por el sistema de 
realidad virtual son trasladados a objetos del propio software,
que son las Observaciones y las Propiedades.

De manera resumida, una Observaci\'on es un hecho interesante en el sistema y que quiere ser observado, y 
est\'a compuesta por propiedades, que son las caracter\'isticas que la definen. En el 
apartado de Antecedentes todo se explicar\'a con mayor detalle, aqu\'i solo se pretende rascar la superficie para 
conseguir una idea general.

Una vez creadas las observaciones con sus propiedades, se crean Situaciones y Pasos. Siendo una Situaci\'on el contexto
en el que suceden las observaciones, y los Pasos c\'omo se comportan ciertas observaciones en el tiempo.

Finalmente, con esos Pasos y Situaciones el sistema, mediante distintas t\'ecnicas de diagn\'ostico 
es capaz de dar un veredicto
a la habilidad que se quiere aprender, e indic\'andole qu\'e ha hecho mal.

\section{Justificaci\'{o}n y prop\'osito}
Actualmente, la selecci\'on de pasos y situaciones se hace manualmente y con un gran componente de intuici\'on,
por lo que se hace evidente la necesidad de un software para facilitar la creaci\'on de Pasos y Situaciones porque
no es una tarea trivial. La creaci\'on de relaciones puede ser muy compleja, y el software MIPS desarrollado
en este TFG
tiene como objetivo simplificar en la medida de lo posible esa tarea, aprovechando al m\'aximo las capacidades
cognitivas del cerebro humano para identificar
patrones complejos. 

Pese a visualizarse gr\'aficamente, la identificaci\'on de esos patrones sigue sin ser algo sencillo. Es
por ello que la tarea principal del software es facilitar la identificaci\'on de intervalos donde se producen los
pasos y situaciones para posteriormente extraer de manera semiautom\'atica las relaciones entre las observaciones y
propiedades que forman parte de cada paso o situaci\'on. Esos pasos y situaciones vienen identificados por el
experto en el dominio que es quien identifica que hechos son relevantes, por ejemplo
``levantar brazo".

\section{Definiciones, acr\'{o}nimos y abreviaturas}
\textbf{TFG:} Trabajo de fin de grado.

\textbf{BD:} Base de datos.

\textbf{CVS:} Control Version System.

\textbf{MVVM:} Model-View-ViewModel.

\textbf{UI:} User Interface. Interfaz de usuario.

\textbf{MVC:} Model-view-controller.

\textbf{IDE:} Integrated Development Environment. Entorno de desarrollo integrado.

\textbf{W3C:} World Wide Web Consortium.
 
%Descripción y situación del trabajo, razones de elección del TFG, 
%planteamiento del problema y justificación del TFG 