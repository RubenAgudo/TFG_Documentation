\chapter{An\'{a}lisis y dise\~{n}o}
%\newpage
%Cómo se plantea el desarrollo del trabajo. En que partes se 
%divide la solución, justificando dicha división y explicando 
%cada una de las partes.

%En aquellos trabajos donde sea aplicable se documentará 
%usando el modelo de dominio, el diagrama de clases, los 
%módulos/paquetes del software desarrollado, etc.

\section{An\'{a}lisis}

\section{Dise\~{n}o}

\subsection{Librer\'{i}as usadas}
\begin{itemize}
    \item \textbf{AvalonDock:} 
    Es una librer\'{i}a que permite crear ventanas acoplables en WPF, al mas puro estilo Eclipse o Visual Studio. Es una librer\'{i}a de
    c\'{o}digo abierto y gratuita, pese a que tambi\'{e}n dispone de una versi\'{o}n profesional que es de pago. La usada en este proyecto
    ha sido la versi\'{o}n gratuita o \emph{community} como dicen ellos. La versi\'{o}n usada esta licenciada bajo licencia BSD.
    \item \textbf{Charting Toolkit}
    Es la librer\'{i}a de Microsoft para realizar distintos tipos de gr\'{a}ficas.
\end{itemize}

\subsection{title}

\subsection{Patrones utilizados}

\subsubsection{MVVM}
MVVM es un patr\'{o}n de dise\~{n}o creado por Microsoft. Intenta conseguir las ventajas de Model-View-Controller como ademas una separaci\'{o}n total
de la vista del controlador. De esta forma los dise\~{n}adores de UI pueden centrar todos sus esfuerzos en crear la interfaz sin preocuparse 
del c\'{o}digo de la aplicaci\'{o}n, ya que se asume que la interfaz va a cambiar mucho durante el ciclo de vida de la aplicaci\'{o}n. Pero,
¿Si separamos totalmente la vista del modelo, como se comunican? En MVC el encargado de eso es el controlador, pero un detalle importante es 
que el c\'{o}digo referente a la vista debe tener llamadas a funciones en el lenguaje utilizado.

Y en este punto es donde la parte VM \footnote{View-Model} toma sentido. el View-Model es una abstracci\'{o}n de la vista, y que es el objetivo 
de los enlaces de datos. El View-Model expone los datos del modelo para que el creador de interfaces pueda enlazar los elementos de la UI
con los datos del modelo (bien utilizando orientaci\'{o}n a objetos o exponiendo los datos de la BD) sin escribir una sola l\'{i}nea de 
c\'{o}digo .NET.

\paragraph{Elementos de MVVM}

\begin{itemize}
    \item \textbf{Model:} Dentro del patr\'{o}n MVC corresponde al modelo de dominio utilizando orientaci\'{o}n a objetos, o a los datos
    representados por una BD.
    \item \textbf{View:} La interfaz de usuario con sus botones, cajas de texto etc.
    \item \textbf{View Model:} Es la \emph{Vista del Modelo}, siendo una abstracci\'{o}n que sirve de mediadora y que expone los datos
    del modelo a la vista, para que esta \'{u}ltima pueda utilizarlas mediante enlaces de datos simples que no requieren de c\'{o}digo, sino
    que se crean mediante XAML. Realmente, no s\'{o}lo expone los datos, si no que ademas convierte los datos del modelo, en datos de vista, listos
    para ser visualizados.
\end{itemize}

\paragraph{Cr\'{i}ticas}
Es curioso que las principales cr\'{i}ticas provengan de su creador, Josh Gossman, el cual dice que utilizar a la fuerza MVVM para operaciones simples
de UI es una exageraci\'{o}n y que si los enlaces de datos si no se gestionan correctamente puede llevar a un consumo excesivo de 
memoria de la aplicaci\'{o}n \cite{MVVM:Criticism}

\subsubsection{Iterator}
Si bien el titulo alude al patr\'{o}n \emph{Iterator} .NET no dispone como Java de la interfaz \texttt{Iterable}.
En cambio, dispone de una interfaz llamada \texttt{IEnumerator<>} que ofrece toda la funcionalidad b\'{a}sica del iterator,
y con algunas funcionalidades extra.

He decidido utilizar este patr\'{o}n por los siguientes motivos:
\begin{enumerate}
    \item \textbf{Abstracci\'{o}n:}
    Al utilizar este patr\'{o}n podemos obtener los elementos de un contenedor, que puede ser un \texttt{array}, una \texttt{LinkedList<>}... sin
    exponer su representaci\'{o}n interna, aumentando la seguridad y previniendo que una clase tenga acceso a cosas que no deba.
    
    \item \textbf{Facilidad de uso:}
    Siempre es preferible utilizar cosas que nos provea el \emph{framework}, ya que no es necesario perder el tiempo en banalidades como 
    recorrer una lista, y probablemente lo haremos menos eficientemente que lo actualmente programado. De todas formas, si necesitamos
    recorrerlo de una manera particular, y que sabemos hacerlo muy eficientemente, no hay problema en no seguir el patr\'{o}n, ya que no son
    normas, si no recomendaciones.
    
    \item \textbf{Prevenci\'{o}n de errores:}
    Relacionado con lo anterior. Si perdemos el tiempo a\~{n}adiendo l\'{i}neas extra la probabilidad de que cometamos un error aumenta. Y aunque
    no lo haya probado emp\'{i}ricamente, tendemos a pensar que \emph{¿Como me voy a equivocar en recorrer un array? ¡Pero si est\'{a} perfecto!}. 
    Y ello nos llevar\'{a} a malgastar m\'{a}s aun nuestro preciado tiempo. Pero al menos nos daremos cuenta de nuestro error y veremos lo
    \'{u}tiles que son los patrones.
\end{enumerate}

\subsection{SOLID}
SOLID es un acr\'{o}nimo mnem\'{o}nico acu\~{n}ado a comienzos de la d\'{e}cada de los 2000 por Robert C. Martin \cite{bringhurst:elements} 
que hace representa cinco principios b\'{a}sicos de la programaci\'{o}n orientada a objetos y el dise\~{n}o de software.

SOLID merece un apartado en esta memoria ya que he intentado cumplir en el mayor grado que me ha sido posible los cinco principios.

Los principios son los siguientes:
\subsubsection{Single responsibility principle}
Traducido al castellano: Princpio de responsabilidad \'{u}nica. El nombre puede llevar a enga\~{n}o ya que Martin define como \emph{responsabilidad}
a una \emph{raz\'{o}n de cambio} \cite{brunetti-JMMM-2002}. Martin, concluye que una clase, debiera tener, una, y solamente una raz\'{o}n para cambiar.

Por ejemplo, imaginemos que disponemos de una clase, o m\'{o}dulo que compila e imprime un informe. Ese informe puede cambiar por dos razones,
de estilo, o de contenido. 

El principio de responsabilidad \'{u}nica dice que una clase solo debiera cambiar por un motivo, y por tanto tendr\'{i}amos que separar el m\'{o}dulo
en dos: El que compila el informe, y el que lo imprime.

Utilizando este principio volveremos nuestro c\'{o}digo mas robusto, ya que un cambio en una clase, no deber\'{i}a afectar a la salida, y por tanto
los siguientes m\'{o}dulos que se alimenten de esa clase.

\subsubsection{Open/Closed principle}
El principio abierto / cerrado implica que una clase est\'{a} abierta a la extensi\'{o}n pero cerrada a la modificaci\'{o}n. Ya que un cambio en el
funcionamiento de una clase har\'{i}a tambalear la estabilidad de nuestro software.

\subsubsection{Liskov's substitution principle}
El principio de sustituci\'{o}n de Liskov \footnote{\url{https://en.wikipedia.org/wiki/Barbara_Liskov}} sostiene que cualquier subtipo de una clase
debe poder sustituir a su supertipo sin que haya un cambio de comportamiento.

M\'{a}s formalmente: Si \textbf{S} es un subtipo de \textbf{T} entonces \textbf{T} puede sustituirse por \textbf{S} y 
conservar su exactitud, tarea que realiza etc.

\subsubsection{Interface segregation principle}
El principio de segregaci\'{o}n de interfaces dice que es mejor tener muchas interfaces espec\'{i}ficas, que una gran interfaz que contenga todo.

De esta manera evitamos que haya partes del software que dependan de m\'{e}todos que no influyen en esa clase.

\subsubsection{Dependency inversion principle}
El principio de Inversi\'{o}n de dependencia dice que:
\begin{enumerate}
    \item Los m\'{o}dulos de alto nivel no deben depender en m\'{o}dulos de bajo nivel. Ambos deben depender de abstracciones
    \item Las abstracciones no deben depender en los detalles. Los detalles deben depender de las abstracciones.
\end{enumerate}
