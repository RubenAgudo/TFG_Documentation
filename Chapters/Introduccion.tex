\chapter{Introducci\'{o}n} 
%\newpage
El Trabajo de Fin de Grado (a partir de ahora TFG) aqu\'{i} presentado fu\'{e} propuesto por un profesor de la universidad.
Las razones de la aceptaci\'{o}n del TFG fueron que se ve\'{i}a una continuidad, es decir, que se pod\'{i}a 
utilizar el TFG como entrada a un grupo de investigaci\'{o}n y/o continuar con el trabajo durante un m\'{a}ster.

\section{Planteamiento del problema}
El problema que el proyecto dise\~nado por el grupo de investigaci\'on GALAN en colaboraci\'on
con la universidad de Navarra intenta resolver, es ense\~nar a alumnos habilidades. Es decir, ayudar a 
instruir a las personas cosas que no son f\'acilmente evaluables mediante un examen escrito.

Para ello se ha dise\~nado un software que permite al alumno capturarse realizando una tarea, y el
software dice si se est\'a realizando bien o no.

Los datos capturados en bruto por el sistema de realidad virtual son trasladados a objetos del propio software,
que son las Observaciones y las Propiedades.

De manera resumida, una Observaci\'on es un hecho interesante en el sistema y que quiere ser observado, y 
est\'a compuesta por propiedades, que son las caracter\'isticas que la definen. De todas formas, en el 
apartado de Antecedentes todo se explicar\'a con mayor detalle, aqu\'i solo se pretende rascar la superficie para 
conseguir una idea general.

Una vez creadas las observaciones con sus propiedades, se crean Situaciones y Pasos. Siendo una Situaci\'on el contexto
en el que suceden las observaciones, y los Pasos como se comportan ciertas observaciones en el tiempo.

Finalmente, con esos Pasos y Situaciones el sistema, mediante distintas t\'ecnicas es capaz de dar un veredicto
a la habilidad que se quiere aprender, dici\'endole al usuario si ha conseguido la habilidad o no.

\section{Justificaci\'{o}n y prop\'osito}
Se hace evidente la necesidad de un software para facilitar la creaci\'on de Pasos y Situaciones porque
no es una tarea trivial. La creaci\'on de relaciones puede ser muy compleja, y el software de este TFG
tiene como objetivo simplificar en la medida de lo posible esa tarea, aprovechando al m\'aximo las capacidades
cognitivas \footnote{\url{https://es.wikipedia.org/wiki/Cognici\%C3\%B3n}} del cerebro humano para identificar
patrones complejos que no son f\'acilmente identificables si no se presentan gr\'aficamente.

Adem\'as, deber\'a ser un software sencillo de utilizar, centrando la interfaz y todos los controles en la tarea principal:
Identificar esos Pasos y Situaciones.

\section{Definiciones, acr\'{o}nimos y abreviaturas}
\textbf{TFG:} Trabajo de fin de grado.

\textbf{BD:} Base de datos.

\textbf{CVS:} Control Version System.

\textbf{MVVM:} Model-View-ViewModel.

\textbf{UI:} User Interface. Interfaz de usuario.

\textbf{MVC:} Model-view-controller.

\textbf{IDE:} Integrated Development Environment. Entorno de desarrollo integrado.

\textbf{W3C:} World Wide Web Consortium.
 
%Descripción y situación del trabajo, razones de elección del TFG, 
%planteamiento del problema y justificación del TFG 