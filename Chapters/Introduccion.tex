\chapter{Introducci\'{o}n} 
%\newpage
El Trabajo de Fin de Grado (a partir de ahora TFG) aqu\'{i} presentado fu\'{e} propuesto por Mikel Villama\~{n}e Giron\'{e}s.
Las razones de la aceptaci\'{o}n del TFG fueron que ve\'{i}a una continuidad, es decir, que pod\'{i}a hacer el TFG como entrada
a un grupo de investigaci\'{o}n y/o continuar con el trabajo durante el m\'{a}ster.

\section{Planteamiento del problema}
El TFG plantea lo que coloquialmente hablando un sistema de telemetr\'{i}a. Previamente, en unas sesiones de captura, se ha capturado
a un experto realizando ciertas acciones, que ser\'{a}n utilizadas posteriormente como una base.

Supongamos la situaci\'{o}n de realizar un \emph{drive} con un palo de golf. La identificaci\'{o}n de los momentos clave de dicho \emph{drive}
a d\'{i}a de hoy no pueden realizarse autom\'{a}ticamente. Por tanto el presente TFG es justo para eso, para permitir de una manera
mas gr\'{a}fica y clara poder seleccionar en que momentos sucede una acci\'{o}n.

\section{Justificaci\'{o}n}

Este TFG est\'{a} justificado por ser simplemente una necesidad que tiene el grupo de investigaci\'{o}n GALAN.

%Descripción y situación del trabajo, razones de elección del TFG, 
%planteamiento del problema y justificación del TFG 