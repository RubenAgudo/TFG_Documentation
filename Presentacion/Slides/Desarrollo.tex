%\section{Herramientas utilizadas}
\begin{frame}
	\frametitle{Desarrollo}
	\framesubtitle{Herramientas utilizadas}
	
	\begin{itemize}
		\item Visual Studio 2013 Ultimate
		\item OxyPlot
		\item AvalonDock
		\item Git
		\item \LaTeX \ y TeXstudio
	\end{itemize}
\end{frame}

\begin{frame}
	\frametitle{Dise\~no}
	\framesubtitle{SOLID y patrones}
	\begin{columns}[T]
		
		\begin{column}[T]{0.5\linewidth}
			\begin{block}{SOLID}
				\begin{itemize}
					\item \textbf{S}ingle responsibility principle
					\item \textbf{O}pen-closed principle
					\item \textbf{L}iskov substitution principle
					\item \textbf{I}nterface segregation principle
					\item \textbf{D}ependency inversion principle
				\end{itemize}
			\end{block}
		\end{column}
		
		\begin{column}[T]{0.5\linewidth}
			\begin{block}{Patrones}
				\begin{itemize}
					\item Model View View-Model
					\item Iterator
					\item Singleton
				\end{itemize}
			\end{block}
		\end{column}
		
	\end{columns}

\end{frame}

%\section{Desarrollo}
\begin{frame}
	\frametitle{Desarrollo}
	\framesubtitle{Qu\'e se ha hecho}
	Una aplicaci\'on que permite:
	\begin{enumerate}
		\item Cargar un XML con las observaciones y propiedades
		\item Visualizar datos discretos y continuos
		\item Cargar v\'ideos y visualizarlos
		\item Seleccionar rangos
		\item Exportar los rangos seleccionados en XML
		\item Organizar los elementos como se desee, tipo Eclipse o Visual 
		Studio
	\end{enumerate}
	
\end{frame}


