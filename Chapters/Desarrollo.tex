\chapter{Desarrollo}
%Explicacion del trabajo desarrollado usando un lenguaje natural
%se debe dar respuesta a cuestiones como ¿Que se ha hecho? Tambien se explicara
%que problemas se han encontrado durante el desarrollo del trabajo, que soluciones 
%se han estudiado, cual fue la elegida etc. Todo ello justificando las decisiones tomadas

\section{Qu\'{e} se ha hecho}
Se ha desarrollado una aplicac\'{i}on que dados unos datos organizados
de una determinada manera permite la visualizaci\'{o}n de los mismos y
la selecci\'{o}n de unos rangos para guardar determinar que en ese rango de tiempo
est\'{a} sucediendo una acci\'{o}n determinada.

Los datos representan Pasos y Observaciones. Cada paso y observaci\'{o}n tiene una serie
de propiedades que son las que se visualizan. Estos datos, cuando fueron capturados, puede
que tengan uno o varios v\'{i}deos asociados. La visualizaci\'{o}n de los v\'{i}deos es completamente
optativa, y en caso de visualizarlos, ir\'{a}n sincronizados con los gr\'{a}ficos.

Es decir, en cada gr\'{a}fico habr\'{a} una l\'{i}nea de progreso para que sea f\'{a}cil determinar en que momento
de simulaci\'{o}n estamos \footnote{Un momento de simulaci\'{o}n es la unidad de tiempo elegida para la captura de datos, no
    necesariamente es un segundo.}.

\section{Decisiones importantes tomadas}
La elecci\'{o}n de la plataforma y entorno de desarrollo ven\'{i}a dada, por lo que las elecciones han sido tomadas
tomando como base que iba a usarse como Sistema Operativo, Windows 8.1 y Visual Studio 2013 como IDE.

\subsection{¿WPF o Windows Forms?}
Ambas tecnolog\'{i}as pertenecen a Microsoft, pero tienen diferencias clave. Entre otras cosas, WPF tuvo su lanzamiento inicial en 2006
junto con .NET Framework 3 \cite{WPF:Overview}. En cambio Windows Forms es una tecnolog\'{i}a 
mucho mas madura, que tiene a sus espaldas 12 años de desarrollo, y que fue lanzada junto con la primera versi\'{o}n de .NET Framework.

\subsubsection{Comparativa}
Lo mejor para realizar la comparativa es mediante una tabla, ya que se ven las capacidades de cada tecnolog\'{i}a.
La tabla ha sido confeccionada a partir de los datos obtenidos de \cite{WPFvsWinForms:Comparative} y elaboraci\'{o}n propia.

\begin{table}[H]
    \begin{center}
        \rowcolors{1}{lightgray}{} %\rowcolors{<starting row index>}{<odd row color>}{<even row color>}
        \begin{tabular}{|p{5cm} | p{4cm} | p{4cm}|}
            \rowcolor{darkgray}                         & \color{white}Windows Forms               & \color{white}WPF \\
            Formularios y controles                     & Si                                       & Si \\
            Documentos en pantalla                      & Si                                       & Si \\
            Documentos de formato fijo (XPS, PDF)       & No                                       & Si \\
            Im\'{a}genes                                & No                                       & Si \\
            V\'{i}deo y audio                           & No                                       & Si \\
            Gr\'{a}ficos 2D                             & No                                       & Si \\
            Gr\'{a}ficos 3D                             & No                                       & Si \\
            Interfaz compatible con altas resoluciones  & No, basada en BMP                        & Si, basada en vectores \\
            Creaci\'{o}n de la interfaz                 & Arrastrando y soltando los elementos     & Interfaz declarativa tipo XML \\
            Multilenguaje                               & Mediante archivos de recursos, f\'{a}cil y bien documentado & Mediante DLLs sat\'{e}lite, poco documentado y las herramientas aun no est\'{a}n listas para un entorno de producci\'{o}n. \\
            \hline
        \end{tabular}
    \end{center}
\end{table}

Como la mayor\'{i}a de las cosas que necesitaba para el proyecto WPF las cumpl\'{i}a sin a\~{n}adidos extra como GDI+ me decant\'{e} por
WPF.

\subsection{Sistema de gr\'{a}ficos}
A la hora de elegir de que manera iba a tratar los gr\'{a}ficos establec\'{i} una serie de caracter\'{i}sticas
que se deb\'{i}an satisfacer para elegir la librar\'{i}a.

Las condiciones son:
\begin{itemize}
    \item Permitir mostrar gr\'{a}ficos cont\'{i}nuos y discretos, y estos \'{u}ltimos tanto num\'{e}ricos como por categor\'{i}as.
    \item Permitir la selecci\'{o}n de rangos con el rat\'{o}n.
    \item Permitir mostrar el progreso del v\'{i}deo asociado. Bien sin programar la funci\'{o}n o al menos que fuese f\'{a}cilmente programable.
\end{itemize}

\subsubsection{Gr\'{a}ficos integrados de WPF}
Al principio parec\'{i}a una alternativa interesante, ya que no habr\'{i}a que depender de librer\'{i}as externas. Pero resulta
que son unos gr\'{a}ficos muy sencillos, con apenas opciones y que b\'{a}sicamente son unas polilineas sobre un elemento canvas.

Para conseguir satisfacer las condiciones requeridas para ser elegido requerir\'{i}a de un trabajo importante construyendo por mi mismo
las funciones requeridas, ya que de serie no cumple mas que el primer punto, adem\'{a}s sin mostrar datos sobre las leyendas ni permitiendo superponer
varias series de datos etc.

\paragraph{Ventajas:} Integrado en WPF, las funciones de serie son muy f\'{a}ciles de usar, documentaci\'{o}n abundante en MSDN.
\paragraph{Desventajas:} De serie no permite la selecci\'{o}n de rangos ni leyendas, ni mostrar el progreso asociado a un v\'{i}deo.
\paragraph{Veredicto:} Rechazado. 

\subsubsection{AForge .NET}
Framework ampliamente recomendada y usada para el tratamiento de im\'{a}genes, gr\'{a}ficos, Computer Vision, inteligencia artificial, machine learning etc.

De serie soporta mostrar los gr\'{a}ficos continuos y discretos y la selecci\'{o}n de rangos. Al permitir la selecci\'{o}n de rangos
intuyo que programar la sincronizaci\'{o}n con el v\'{i}deo es bastante f\'{a}cil.

\paragraph{Ventajas:} Framework Open Source, cumple con todos los requisitos, documentaci\'{o}n abundante.
\paragraph{Desventajas:} Los controles para mostrar los gr\'{a}ficos son controles .NET, no WPF. La documentaci\'{o}n sobre mostrar los gr\'{a}ficos no
esta demasiado elaborada.
\paragraph{Veredicto:} Rechazado.

\subsubsection{WPF Toolkit}
Extensi\'{o}n open source de las capacidades de mostrar gr\'{a}ficos de WPF desarrollado por Microsoft de manera oficial. 
Ofrece una manera muy simple y muy clara depoder enlazar datos con un ViewModel o bien de manera program\'{a}tica. 
Admite la representaci\'{o}n de todo tipo de gr\'{a}ficos imaginables: Histrogramas, \'{a}rea, barras horizontales, 
l\'{i}nea, dispersi\'{o}n, circular (tarta), burbuja...
 
La presentaci\'{o}n de los gr\'{a}ficos es muy profesional, pero no he encontrado ninguna manera de poder seleccionar un rango sin tener que modificar
por mi mismo las clases del propio toolkit. Por lo que \'{u}nicamente cumple uno de los tres requisitos indispensables, y el \emph{workaround} no
es algo que se haga en un momento y no es algo sencillo de llevar a cabo.

\paragraph{Ventajas:} Open Source, mantenido por Microsoft, por lo que podemos asegurar una estabilidad alta, sencillez de uso, gr\'{a}ficos profesionales.
\paragraph{Desventajas:} Ausencia total de documentaci\'{o}n, el proyecto lleva sin actualizarse desde 2010, no permite de manera f\'{a}cil seleccionar 
rangos ni sincronizar con v\'{i}deo.
\paragraph{Veredicto:} Rechazado.

\subsubsection{OxyPlot}

\section{Problemas}