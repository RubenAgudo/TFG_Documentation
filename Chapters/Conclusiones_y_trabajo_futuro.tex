\chapter{Conclusiones y trabajo futuro}

\section{Mejoras}
Utilizar bien SOLID

Cambiar los Singleton por patrones Factory para cumplir correctamente SOLID

Implementar pruebas unitarias

Intentar utilizar los enlaces de datos que proporciona .NET

Mejorar la manera en la que se utilizan los workers.

Que cuando se pinche en un elemento que ya esta abierto, que consiga el foco

Los graficos de cada observacion que tengan distintos colores para poder asociar
visualmente un color a cada observacion

Utilizar un modulo de mineria de datos, ya que muchas veces las observaciones
seran muy similares entre si por lo que podria utilizarse para identificar 
rangos automaticamente, y que el usuario diga si estan correctos o no.

Eliminar los timers de la aplicacion, ya que es mucho mas eficiente utilizar
INotifyPropertyChanged.

Convertir MainWindow en UserControl para que pueda se añadido a otro dockingmanager.
Realmente esto es muy facil, solo habria que crear un userControl llamandose Workbench
con todo el XAML y el codigo cs de MainWindow y despues una aplicacion en la que su
componente principal sea ese UserControl

Crear un dialogo de opciones para personalizar el comportamiento de la aplicacion

\section{Conclusiones}
Hay que contar como han sido realmente las cosas: tiempos, distribucion del trabajo, 
sensaciones que se han tenido al usar scrum etc. Por lo que habra que crear el
burndown para mostrar como ha sido realmente el trabajo, que nuevas tareas han surgido...

Algo sesudo, como por ejemplo, el sindrome del programador, que siempre se tienen nuevas
ideas para implementar, ideas no contempladas desde el principio pero que no aportan nada.
Por ejemplo, cada intervalo que se crea, poder mostrarlo en algun sitio y que se permita
el borrado de los intervalos previamente creados, editarlo etc.

Lo complicado que es trabajar en una tecnologia que apenas conoces, ademas utilizando
librerias con escasa documentacion, en la que todo tu trabajo se basa en prueba y error
y en mirar si otras personas han tenido antes el mismo problema que tu utilizando
ademas la misma biblioteca.