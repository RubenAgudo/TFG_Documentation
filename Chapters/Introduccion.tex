\chapter{Introducci\'{o}n} 
%\newpage
El Trabajo de Fin de Grado (a partir de ahora TFG) aqu\'{i} presentado fu\'{e} propuesto por Mikel Villama\~{n}e Giron\'{e}s.
Las razones de la aceptaci\'{o}n del TFG fueron que ve\'{i}a una continuidad, es decir, que pod\'{i}a hacer el TFG como entrada
a un grupo de investigaci\'{o}n y/o continuar con el trabajo durante el m\'{a}ster.

\section{Planteamiento del problema}
El TFG plantea lo que coloquialmente hablando es un sistema de telemetr\'{i}a. 
Previamente, en unas sesiones de captura, se ha capturado
a un experto realizando ciertas acciones, que ser\'{a}n utilizadas posteriormente como una base.

Supongamos la situaci\'{o}n de realizar un \emph{drive} con un palo de golf. La identificaci\'{o}n de los momentos clave de dicho \emph{drive}
a d\'{i}a de hoy no pueden realizarse autom\'{a}ticamente. Por tanto el presente TFG es justo para eso, para permitir de una manera
mas gr\'{a}fica y clara poder seleccionar en que momentos sucede una acci\'{o}n.

\section{Justificaci\'{o}n}

El software se lleva a cabo porque uno de los integrantes del grupo de investigaci\'{o}n me lo propuso como TFG. (Porque esta justificado,
Mikel?)

\section{Prop\'{o}sito}
Ofrecer un software de calidad que cumpla con todos los requerimientos y expectativas del grupo de investigaci\'{o}n.
As\'{i} mismo ser\'{a} un software potente pero sencillo de utilizar adem\'{a}s de disponer de un apartado gr\'{a}fico agradable.

\section{Definiciones, acr\'{o}nimos y abreviaturas}
\textbf{TFG:} Trabajo de fin de grado.

\textbf{BD:} Base de datos.

\textbf{CVS:} Control Version System.

\textbf{MVVM:} Model-View-ViewModel.

\textbf{UI:} User Interface. Interfaz de usuario.

\textbf{MVC:} Model-view-controller.

\textbf{Observaci\'{o}n:} Una observación describe un hecho que puede tener lugar en el Sistema interactivo 
durante un intervalo de tiempo. Viendo esta definición, hay que destacar que una
observación describe un hecho y que puede tener lugar durante un intervalo de
tiempo.

Cada elemento de observación está compuesto por una propiedad que puede ser de
distinto tipo: número, texto, objeto… Por ejemplo, podemos tener la observación
Vehículo que contendría la propiedad Velocidad, de tipo double. \cite{INTRASIM:manual}

\textbf{IDE:} Integrated Development Environment. Entorno de desarrollo integrado.

\textbf{W3C:} World Wide Web Consortium.
 
%Descripción y situación del trabajo, razones de elección del TFG, 
%planteamiento del problema y justificación del TFG 