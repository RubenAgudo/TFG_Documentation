\chapter{Antecedentes}

%Situación actual, estudio de diferentes alternativas 
%existentes o distintos posibles enfoques del problema a solucionar
\section{Situaci\'on actual}
El grupo de investigaci\'on GALAN, esta construyendo y manteniendo un sistema de simulaci\'on que sirve para
apoyar en la ense\~nanza a los alumnos, sobre todo en cosas que tenga que ver con actividad f\'isica.

Actualmente, cuando capturan al experto realizando el ejercicio que se ha definido, no saben a ciencia cierta
cuando est\'a sucediendo que cosa.

Por ejemplo, si est\'an capturando un \emph{drive} de golf, que se compone de distintas fases, no saben exactamente
cuando empieza realmente la acci\'on de bajar los brazos para empezar el golpeo, por lo que lo hacen ``a ojo´´.

Al ver que este m\'etodo es poco preciso y adem\'as poco conveniente ya que se trata un poco de prueba y error,
se me propuso realizar una aplicaci\'on que llevaban tiempo ideando.

\section{Estudio de diferentes alternativas}
La aplicaci\'on consiste en poder visualizar tanto en gr\'aficos como en v\'ideo lo capturado por Kinect. Al visualizar
los gr\'aficos por pantalla y adem\'as sincronizados con uno o varios v\'ideos puede elegirse con mayor precisi\'on.

Para conseguir este objetivo, se barajaron distintos tipos de acercamientos al problema.

\subsection{M\'ultiples ventanas dentro de una ventana maestra (MDI)}
Al principio lo primero que se me ocurri\'o fue crear una interfaz MDI (Multiple Document Interface). Como ventajas
ten\'ia que ya hab\'ia creado algunas aplicaciones MDI en .NET por lo que no habr\'ia mucho problema cre\'andola.

Pero afront\'emoslo, estamos en 2014 y las ventanas MDI no son muy c\'omodas de utilizar cuando se disponen de varias ventanas
abiertas, ya que no se acoplan autom\'aticamente a los lados, ni se pueden crear pesta\~nas de manera din\'amica.
Por otro lado, las ventanas MDI fueron desaprobadas en WPF, que es el nuevo sistema de ventanas de Microsoft.